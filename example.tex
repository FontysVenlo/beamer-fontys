\documentclass[aspectratio=169]{beamer}

% Needed packages
\usepackage{hyperref}
\usepackage{listings}
\usepackage{tabularx}

% Use the Fontys theme
\usetheme[lang=en]{Fontys}



% Title page settings
\title{Example presentation for testing the Fontys theme}
\institute{FHTenL}
\author{M. Bonajo}

% Start the document
\begin{document}

% Create title page
\begin{titleframe}
    \titlepage
\end{titleframe}

\begin{frame}
    \tableofcontents
\end{frame}

\section{Different lists}
% Example frame for itemize
\begin{frame}
    \frametitle{Frame showing itemize}

    \begin{itemize}
        \item Option one
        \begin{itemize}
            \item Suboption one
            \item Suboption one
        \end{itemize}
        \item Option two
        \item Option three
    \end{itemize}

\end{frame}

\begin{frame}
    \frametitle{Frame showing enumerate}

    \begin{enumerate}
        \item Option one
        \begin{enumerate}
            \item Suboption one
            \item Suboption two
        \end{enumerate}
        \item Option two
        \item Option three
    \end{enumerate}
    \EN{English text here}
    \DE{German text here}
\end{frame}

\begin{frame}
    \frametitle{Frame showing description}

    \begin{columns}
        \begin{column}{0.5\textwidth}
            Default alignment
            \setbeamertemplate{description item}[default]
            \begin{description}%
                \item[One] Option one%
                \item[Two] Option two%
                \item[Three] Option three%
            \end{description}%
        \end{column}
        \begin{column}{0.5\textwidth}
            Left alignment
            \setbeamertemplate{description item}[align left]
            \begin{description}%
                \item[One] Option one%
                \item[Two] Option two%
                \item[Three] Option three%
            \end{description}%
        \end{column}
    \end{columns}

\end{frame}

\section{Example of URL}
% Example frame for URL
\begin{frame}
    \frametitle{Frame showing an URL}
    
    You can find more information here \href{http://www.example.com}{Example link}

\end{frame}

\section{Example of custom frametitle}
\begin{titleframe}
    \frametitle{Frame showing you can have multiple titleframes}
    Lorem ipsum dolor sit amet, consectetur adipiscing elit, sed do eiusmod tempor incididunt ut labore et dolore magna aliqua. 
\end{titleframe}

\section{Example of graphics}
\begin{frame}
    \frametitle{Frame showing graphics}
    
    \begin{figure}
        \centering
        \includegraphics[width=0.8\textwidth]{logo_purple_society_en.png}
        \caption{The Fontys logo with for society used in English communication}
        \label{fig:logo}
    \end{figure}

\end{frame}

\section{Example of table}
\begin{frame}
    \frametitle{Frame showing table}

    \begin{table}
        \begin{tabularx}{\textwidth}{ |X|X|X|X| }
            \hline
            label 1 & label 2 & label 3 & label 4 \\
            \hline 
            item 1  & item 2  & item 3  & item 4  \\
            \hline 
            item 1  & item 2  & item 3  & item 4  \\
            \hline 
            item 1  & item 2  & item 3  & item 4  \\
            \hline
          \end{tabularx}
          \caption{Example table}
    \end{table}

\end{frame}

\section{Example of source code}
\begin{frame}[containsverbatim]
    \frametitle{Frame showing code in lstlisting}

\begin{lstlisting}[language=Java]
static int gcd( int a, int b ) {
    // make sure params to computation are positive.
    if (a < 0){
        a = -a;
    }
    if (b < 0){
        b = -b;
    }
    while(b != 0){
        int t = b;
        b = a % b;
        a = t;
    }
    return a;
}
\end{lstlisting}

\end{frame}

\end{document}